%        File: pnas1.tex
%     Created: Wed Apr 30 04:00 pm 2014 E
% Last Change: Wed Apr 30 04:00 pm 2014 E
%
\documentclass[a4paper]{report}
\usepackage[hmargin=1in,vmargin=1in]{geometry}

\usepackage{graphicx}

\usepackage{amsmath}
\usepackage{amssymb}
\usepackage{mathrsfs}
\usepackage{gensymb}
\usepackage{algorithm2e}
\usepackage{amsthm}

\newtheorem*{mydef}{Definition}

\usepackage{cite}

\usepackage{framed, color}
\usepackage{soul}
\usepackage[colorlinks=false, urlcolor=blue]{hyperref}

\newcommand{\td}[1]{{\color{blu}\hl{TODO: #1}}}

\definecolor{shadecolor}{rgb}{.93,.93,.93}
\definecolor{blu}{rgb}{0,0,1}
\setlength{\parindent}{0cm}
\setlength{\parskip}{4mm plus1mm minus1mm}


\title{The effects of priming in human computation}
\author{Edward Newel, Derek Ruths}
 
\parindent0pt \parskip8pt

\begin{document}
\maketitle
\section*{Abstract}
\section*{Significance Statement}
\section*{Introduction}

Microtask crowdsourcing platforms like Amazon Mechanical Turk (AMT) make it 
possible to submit batches of small tasks to a large pool of human workers, 
who do the tasks for fun and remuneration \cite{kazai2013analysis}.  Tasks 
involving qualitative
judgment (e.g. flagging inappropriate images), natural language annotation,
and object recognition are commonplace, since people still outperform  
computers in these areas \cite{yuen2011survey}.

Such platforms render human cognition as fluid a resource as compute cycles: 
it is available on demand in small units, with low transaction cost, from 
anywhere via the Internet \footnote{\href{http://mturk.com}{http://mturk.com}}.
The task requester can interact with the platform like a compute server, 
seamlessly 
integrating human and machine computation.  Researchers have put forward the 
term HPU (Human co-Processing Unit)\footnote{We acknowledge the irony of this 
term: the original meaning of `computer' was a person who performs 
computations, its reference to machines being a personification brought into 
usage in the late 1800s\cite{Dictionary:hl}.}, viewing the introduction of 
microtask platforms as a fundamentally new computing architecture
\cite{5543192}.  

Originally used to distribute clerical work, these platforms 
increasingly serve as a cheap and fast means to engage experimental 
participants and provide qualitative annotations of datasets in a research 
setting \cite{snow2008cheap}.  Given the limited ability to screen workers, and their transient 
low-commitment engagement with the requester, researchers have criticized the
reliability of crowdsourced output \cite{marsden2009crowdsourcing}.  

For one, cheating
is a problem \cite{lease2011quality}, but straightforward methods have been 
developed to screen out cheaters \cite{snow2008cheap, kazai2013analysis}.
Still, there is a great deal of variance in worker attention and 
accuracy \cite{kazai2013analysis}.  Researchers have found many factors that 
modulate worker performance, such as level of pay\cite{kazai2013analysis}, 
training\cite{le2010ensuring}, pre-screening 
workers\cite{paolacci2010running}, and user-interface design
\cite{Finnerty2013}.

Here we are concerned with effects of a more subtle origin: the influence that
prior tasks have on subsequent ones.  Psychology experiments document the 
strong effects that a prior exposure to stimulus can have on participants' 
performance in related tasks, a phenomenon known as \textit{priming}[***].  
Strong 
priming effects have been demonstrated for
perceptual tasks such as the visual recognition of 
objects \cite{BJOP:BJOP1796},
auditory recognition of words \cite{BJOP:BJOP1826}, and the recognition
of gender[***] and affect[***]: precisely the kinds of tasks well-serviced by 
microtask platforms \cite{yuen2011survey, snow2008cheap}.  

In the direction of priming, researchers have investigated various effects 
due to \textit{framing}, by disclosing to the worker the functional role of
their task in the workflow context \cite{Kinnaird2012281}, 
or describing a meaningful ultimate purpose of the project in which their
task is a part \cite{chandler2013breaking}.
However, psychological experiments show that priming is strongest when 
delivered through the same modality as the task [***].  This leads us to 
investigate how priming that occurs \textit{during} a task influences output
of subsequent similar tasks. We call such effects \textit{in-task priming}.  

We investigate the effects of in-task priming for an image-labeling task
on the AMT platform.  Each worker is required to label a \textit{prior set} of
images followed by a \textit{posterior set}, but no distinction is made between
these sets from the perspective of the worker.  The posterior set is
always the same, but workers are randomly assigned one of three prior sets.
We analyze the labels workers attribute to the posterior set to see if it is 
affected by the prior set.  As effects, we consider what aspects of image 
content the workers choose to label (the \textit{label orientation}), and how 
specific the labels are (the \textit{label specificity}), both of which have 
direct consequence on the validity of using human computation to make 
qualitative judgments.

As a point of comparison, we subject some groups of workers to a kind of
framing: we disclose a fictitious, semantically-loaded name for the 
task-requester, chosen to signal an interest in a specific aspect of the 
image content.  One might expect that this would strongly sway the labeling 
of the workers.  We do detect the effects of framing, but, remarkably, we find 
that in-task priming has a much stronger effect on label orientation and 
specificity.  

This has major implications for the design of qualitative labeling and 
classification tasks.  Our results show that prior tasks can introduce 
significant bias in subsequent worker labeling.  On the other hand, our 
results suggest that the progression of tasks can also be used to tune 
workers to produce more nuanced output.

In psychology, priming is defined as the effect in which a prior stimulus leads
to faster or more accurate responses in a subsequent task than without the
stimulus, or where the stimulus allows the experimenter to the elicit 
responses from subsequent stimuli that are less intense or noisier [***].
However priming might have many effects other than those traditionally 
measured.  We are interested in studying the effects of priming in a human 
computation, we adopt a more general view of priming as HPU hysteresis.

For this purpose, we present an algorithmic definition of priming that
yields a powerful method for detecting priming without knowing ahead of time 
what effects one is looking for.  We demonstrate that this tool is capable of 
distinguishing the priming treatments in our experiment even when the 
differences are subtle.  We expect such a tool will have wide applications 
outside computer science.

\subsection*{An algorithmic definition of priming}

One difficulty with the notion of priming is that there appears to be no 
defensible notion of an ``unprimed state''.  A participant always approaches
a task in \textit{some} state of mind.  This will be influenced by her
exposures prior to the experiment, which cannot be known by the experimenter.  
In our experiment, for example, we include a prior image set that is in some
sense ``neutral'', but it is by no means a control group. The particular 
images used would arguably induce different priming than that induced by 
another set that was, in its own way, ``neutral''.
We avoid this difficulty by defining priming relatively, and speak of one
treatment being \textit{differently primed} from another.

Another factor that we must consider is that the effects of priming will
manifest differently depending on the task that participants perform 
subsequent to priming.  Said differently, two priming treatments that
produce markedly different output for one task might be 
indistinguishable under another task.  Thus, we define priming
in relation to the task.

A final factor to consider is that, individuals who are in the same priming 
treatment will still in general produce different output for a task,
due to the natural variance between people.  Therefore, we define priming in 
relation 
to populations, not individuals.

With these considerations, consider two populations of people 
$\mathcal{J}$ and $\mathcal{K}$, that produce respectively two populations of 
outputs $\mathcal{L}$ and $\mathcal{M}$ during the completion of a task 
$\mathcal{T}$: 

\vspace{2mm}
\begin{mydef}
	\upshape
	$\mathcal{J}$ and $\mathcal{K}$ 
	\emph{differ in priming by a degree at least $\theta$},
	if there exists an algorithm $\mathcal{A}$ that takes as input 
	$L \subset \mathcal{L}$ and $M \subset \mathcal{M}$ and the output 
	$o$ of a person $p$ chosen with equal probability from either 
	$\mathcal{J}\setminus J$ or $\mathcal{K}\setminus K$, 
	such that $\mathcal{A}$ determines the set from which $p$ was drawn
	($\mathcal{J}$ or $\mathcal{K}$) with accuracy $\frac{1+\theta}{2}$.  
\end{mydef}

In the language of computer science, $\mathcal{A}$ is a classifier.  Thus,
intuitively, if there exists a classifier that can distinguish two populations,
then they are differently primed.

We leave the details of $\mathcal{A}$ unrestricted.  This is 
intentional since we want the definition to depend on the inherent 
potential for two populations to be distinguished, rather than on the 
performance of a particular classifier.

A consequence of this choice is that empirical measures of $\theta$ always
represent lower bounds on the theoretical value appearing in the definition.
As a result, a measured $\theta$ that is greater than zero directly 
supports the conclusion that two populations are differently 
primed, up to the statistical significance of the measurement.  However, if 
$\theta$ is not found to be statistically different from zero, one cannot 
directly
conclude that two populations are the same in priming.  The definition
admits the possibility that there exists some other $\mathcal{A}'$ capable of 
generating a measure of $\theta$ statistically greater than zero. 
Interpreting a result
that $\mathcal{A}$ fails to achieve accuracy significantly better than 
chance depends on the capabilities of $\mathcal{A}$ and the state of the art
of classifier algorithm design.

In practice we do not anticipate serious difficulty in choosing a classifier
and establishing a meaningful value of $\theta$.  One of the simplest 
classifier algorithms, the \textit{Naive Bayes classifier}, performs very 
nearly as well as the most sophisticated classifiers in nearly all 
applications [**].  Therefore, we submit $\theta_\text{NB}$, as determined by 
a Naive Bayes classifier, to be a general-purpose measure of priming 
difference.  



\subsection*{Experimental Setup}
We solicited 900 AMT workers to perform an image labeling task.  The workers
were randomly assigned to one of seven treatments shown in 
\textbf{Table 1}.  Workers were first shown brief instructions.
Depending on their treatment, they were then either shown the name of 
a research funder followed by a prior image set, or they were immediately 
shown a prior image set. When shown, the funder name was either
``The Global foundation for Cultural Recognition'', or 
``The National Foundation for Nutritional Awareness''.  Finally, workers were 
presented the posterior image set, which was the same for all treatments.

The prior and posterior images sets both consist of five images. Workers had
to provide five labels for each image.  For the posterior image set we chose 
images that contain prepared meals and feature a prominent, identifiable 
culture (see \textbf{Fig. 1}).  We intended that the content of these images 
would elicit labels focusing food or culture (or both).

One of the prior image sets, which we call \textit{ambiguous}, was chosen to
be very similar to the posterior image set, in the sense that it consists of
images of prepared meals.  However, the ambiguous image set was chosen such 
that its cultural features are less prominent and more difficult to identify 
(see \textbf{Fig. 2}).  
The \textit{cultural} prior set features iconic cultural scenes, but no food 
at all (see \textbf{Fig. 3}).  Finally, images from
the \textit{ingredients} prior image set depict separated ingredients, but, 
like the ambiguous set, were chosen to avoid prominent cultural features (see 
\textbf{Fig. 4}).  See the \textbf{Materials and Methods} section for further 
details.

\subsection*{Results}
\paragraph{The priming of all treatments was easily detectible.}
We test for priming effects using our definition proposed in the introduction,
implemented with a Naive Bayes classifier.  \textbf{Figure 5} shows
the classifier performance in distinguishing between different pairs of 
treatments based on workers' labels for the posterior images, and the corresponding
value for $\theta_\text{NB}$.

The classifier distinguishes the various treatments from one another (pairwise)
with high accuracy, confirming that both priming modalities 
(in-task and framing) were effective.  Remarkably, the classifier not only 
distinguishes treatments tending toward different ``orientations'' 
(e.g. $\textsc{cult}_{img}$ and $\textsc{ingr}_{img}$, \textbf{Fig. 5A}), 
it also distinguishes workers from treatments of the \textit{same} 
orientation with high accuracy 
(e.g. $\textsc{cult}_{img}$ and $\textsc{cult}_{fund, img}$, \textbf{Fig. 5B}).
Clearly the algorithmic definition of priming and its implementation as 
$\theta_\text{NB}$ provides a sensitive means for detecting the presence of
primed populations. 

Next we look at the degree to which labels from individual posterior images reveal 
priming.  It would stand to reason that, as workers proceed through 
the posterior images, the effects of priming might be ``washed out''.  
\textbf{Figure 6} shows the
performance of the Naive Bayes classifier when using the labels workers provide
for particular images.  This does show that, as we look to later posterior images, 
the treatments' difference in priming, $\theta_\text{NB}$, declines.  In future
work, we plan to investigate the dynamics of priming wash-out in greater 
detail.

\paragraph{Priming orients focus.} Having demonstrated differential 
priming, we investigate the \textit{nature} of the priming.  To this
end, we construct an ontology of all the labels attributed to
the posterior images.  The ontology is a directed acyclic graph of terms,
with directed edges drawn from abstractions toward instances.  For example,
our ontology contains the path \texttt{food} $\to$ \texttt{ingredients} $\to$ 
\texttt{vegetables} 
$\to$ \texttt{tomato}. We provide more detail about the construction of the 
ontology in the \textbf{Materials and Methods} section.

The ontology allows us to algorithmically count the number of labels that are
of cultural orientation (\texttt{ganesha}),  food orientation
(\texttt{bread}), or both (\texttt{dim sum}).  We do this by counting the
number of labels having have inbound paths leading from the abstract clasess 
\texttt{culture} and/or \texttt{food}, which are present in the ontology.
\textbf{Figure 7} plots the composition of labels (fraction of culture- and 
food-oriented labels) for different treatments and images.  In 
panel \textbf{7A}, we can see that, before considering the effects of 
treatments, there is significant variance in the composition of labels 
for different images, probably due to inherent differences in their content.  
For our purpose, we are interested in the systematic differences between 
treatments.

In \textbf{Fig. 7B} we exhibit the composition of labels for the various 
treatments (aggregating together the labels for all images of a given 
treatment).  We can see that the treatments which expose workers to the
cultural image sets do elicit significantly more culture-oriented and fewer
food-oriented labels compared to the other treatments ($\alpha=0.05$).  It 
also appeas that all the $\textsc{ingr}_x$ treatments produce more 
food-oriented labels, but we cannot assert this at significance 
$\alpha = 0.05$. 

The effect size, however, is somewhat small.  Since we observed that priming 
difference ``wears off'', it may be the case 
that aggregating label compositions for all images actually masks the effects. 
In \textbf{Fig. 7C}, we plot label compositions for only the first posterior
image.  This reveals that, in labeling the first posterior image, workers in 
$\textsc{cult}_{img,x}$ were strongly influenced to favor culture- over 
food-oriented labels.  The effect was strong enough to essentially reverse 
the proportions of these labels compared to the other treatments.  
This shows that in-task priming effects can dramatically distort the results 
of human computation tasks.

We next inspect this effect longitudinally, that is, as a function of 
the posterior image.  To do so, we compute the \textit{excess cultural orientation} 
($\Delta_{cult}$) of
$\textsc{cult}_{img}$ relative to $\textsc{ambg}$ as follows: 
\begin{align}
	\Delta_{cult}(i) = \frac{1}{N}\left[ \sum_{w\in\textsc{cult}_{img}} \left(N_{w,cult}^{(i)} - N_{w,food}^{(i)}\right)
	- \sum_{w\in\textsc{ambg}} \left(N_{w,cult}^{(i)} - N_{w,food}^{(i)}\right)\right],
\end{align}
where $N_{w,cult}^{(i)}$ stands for the number of culture-oriented labels 
attributed by worker $w$ to image $i$, while $N_{w,food}^{(i)}$ similarly 
counts food-oriented labels, and $N$ is the total number of labels in a 
treatment.  
Intuitively, this is intended to capture how much 
more culture-oriented $\textsc{cult}_{img}$ is than $\textsc{ambg}$.  
\textbf{Figure 8} plots $\Delta_{cult}$ as a function of posterior image, $i$. 
This clearly shows the strong influence on the labels attributed to the first 
posterior image,
but for later images the effect immediately drops below statistically 
significant levels, while remaining always positive.

\paragraph{Priming affects attention to detail.} Using the ontology described
in the previous section, we compare treatments in their degree of specificity.
When comparing two labels $\ell_1$ and $\ell_2$ produced by different workers,
we say that $\ell_2$ is more specific than $\ell_1$ if there is a 
\textit{directed path} from $\ell_1$ to $\ell_2$.  If there is no directed path
between $\ell_1$ and $\ell_2$, then they are non-comparable.  As an example,
\texttt{tomato} is more specific than \texttt{food}, while \texttt{statue}
and \texttt{food} are non-comparable.

We can compare the specificity of two workers by comparing, for each image,
the labels given by one worker with those of the other.  Thus, we define the
specificity of worker $u$ relative to worker $v$ to be $s(u,v)$:

\begin{align}
	s(u,v) = \sum_{i=1}^5 \;
	\sum_{\ell \in u(i) } \;
	\sum_{m \in v(i)} 
	\left(\mathbf{1}_{[\ell > m]} - \mathbf{1}_{[m>\ell]}\right),
	\label{eq:worker-specificity}
\end{align}
where $u(i)$ denotes the set of labels attributed by worker $u$ to image $i$, 
and $\mathbf{1}_{[\ell > m]}$ evaluates to 1 if $\ell$ is more specific than 
$m$, and 0 otherwise.

We can then define the relative specificity of two treatments, 
$S(\mathcal{U},\mathcal{V})$, to be the average
relative specificity of workers drawn uniformly from the treatments.  
We obtain an estimator, $\hat{S}({\mathcal{U}, \mathcal{V}})$, using the 
sample mean:

\begin{align}
	\hat{S}(\mathcal{U},\mathcal{V}) = 
	\frac{1}{|\mathcal{U} \times \mathcal{V}|}
	\sum_{(u,v) \; \in \; \mathcal{U} \times \mathcal{V}} \;
		s(u,v),
		\label{eq:specificity}
\end{align}

where $\mathcal{U}$ and $\mathcal{V}$ are treatments, taken as sets of workers.
See \textbf{Materials and Methods} for statistical analysis of the relative 
specificity of two treatments. 
\textbf{Figure 9} 
exhibits the results.

In panel \textbf{9A}, we see that when we compare each of the treatments to
$\textsc{ambg}$, all treatments in which a non-ambiguous prior image set was 
displayed have \textit{lower} specificity (though not all at $\alpha=0.05$).  
In panels \textbf{9D} and \textbf{9G} we compare the same pairs of treatments, 
but restrict to words of culture- and food-orientation respectively (words
of both orientations are excluded).  These plots show that the relative lack 
of specificity is mainly attributable to food-oriented words.  

Likewise, in panels \textbf{9B}, and \textbf{9C}, 
wherever we see significant differences in specificity, it turns out that
these differences appear to mainly be due to differences in food specificity,
as can be seen by comparing to panels \textbf{9E}, \textbf{9F}, \textbf{9H}, 
and \textbf{9I}.  In fact, looking to panels \textbf{9D}, \textbf{9E}, and 
\textbf{9F},
there are no significant differences in cultural specificity to speak of. 
Clearly, food and culture are not on equal footing in the experiment.  Various 
psychological explanations are possible, and we explore these in the 
discussion below.

\subsection*{Discussion}

\paragraph{In-task priming is strong.}  We expected that the funder names 
would overtly signal to workers that the requester was interested in a 
specific aspect of image content.  On the other hand, we did not 
expect the prior image set, though enriched in certain content, to overtly signal interest.  In turn, we expected that a signal of intent would 
strongly affect how the workers approach the task.  Perhaps the most 
unexpected result of this study is that the prior image set influenced label 
orientation and specificity far more than did framing the task by naming the
funder.

This points to a serious hazard in conducting studies using human computation,
academic or otherwise: even if the requester has 
eliminated surrounding influences to every practical extent, 
\textit{the greatest source of bias might lurk in the task itself}.

Randomized ordering and replication of tasks is one method to abate this.
But the requester could also include an initial set of tasks as a 
``tuning phase''.  During the tuning phase, workers would be primed in a 
controlled way, through exposure to tasks that had been specially picked to
orient worker focus in a useful manner.  Moreover, this phase could double as a
screening phase to select quality workers [***].

\paragraph{The rate of priming wash-out depends on how it is measured}
Seeing that $\theta_\text{NB}$ declines as one proceeds through the test 
images (\textbf{Fig. 6}), one should ask whether this actually represents 
priming wash-out.  One might also attribute this to 
differences in the \textit{content} of the images. It is interesting to 
compare this to the result in \textbf{Fig. 8}, which
shows that the effect of exposure to the prior image set on 
\textit{orientation} (culture vs. food) certainly
decays much more quickly than the overall effects detected by
the classifier.  This appears to be consistent with the notion that the 
mechanisms of priming are multiple and complex [***].  Here we again see an
advantage of using $\theta_\text{NB}$ to measure priming: it is able to detect 
differences in priming even after the expected effects have worn off.

\paragraph{Subtle priming effects can be detected with $\theta_\text{NB}$.}
While the Naive Bayes classifier is able to distinguish all treatment pairs,
many such pairs cannot be distinguished on the basis of label orientation and 
specificity 
(for example $\textsc{cult}_{fund}$ and $\textsc{ingr}_{fund}$).
One might expect, then, that $\theta_\text{NB}$
for the pair $\textsc{cult}_{fund}$ and $\textsc{ingr}_{fund}$, would be 
smaller than for, say, $\textsc{cult}_{img}$ and \textsc{ambg}.  
\textbf{Figure 5A} shows that this is not the case.  Thus, priming can be 
strong without generating the effects one might expect, and in many 
circumstances, such effects might go entirely undetected.  We believe this 
is a strong argument in support of using $\theta_\text{NB}$ to determine 
whether priming has occurred, without making narrow assumptions about what the 
effects of priming will be.

\paragraph{Priming did not affect cultural specificity.}
It was quite surprising that the effects on workers' specificity was 
essentially
confined to food-oriented labels, and that we observed no significant movements
in cultural specificity.   This might reflect the different roles
played by food and culture in the human lexicon.  This seems plausible when we 
consider the vital role of food on the one hand, and the bias in social 
cognition known as \textit{outgroup homogeneity}\cite{Vala2012491,Ray2012387,Ishii2011159,Boldry2007157}.
Outgroup homogeneity is a phenomenon whereby people perceive fewer 
differences between members of an outgroup than between members of their 
ingroup \cite{Boldry2007157}.  One manifestation is that people 
spend less time observing outgroup members when asked to make a character 
judgment \cite{Vala2012491}.  This phenomenon can be explained in terms of 
adaptiveness: social cognition regarding one's ingroup is more consequential
\cite{Vala2012491}.  Thus, it seems plausible that a nuanced cultural 
vocabulary might be of limited advantage.  On the other hand,
the differences between food and poison can be nuanced indeed, and so it would
not be surprising to find that people have a more nuanced vocabulary for 
food than for culture.

Another explanation is that food is more so-called \textit{imageable} than 
culture, because although aspects of culture can be depicted in an image,
culture itself cannot.  Studies show that the brain processes high- and 
low-imageable words differently \cite{Swaab200299}, and it would not be 
surprising if the effects of priming on image labeling were modulated by how 
imageable the labels are.

Even with the above explanations, is surprising that 
$\textsc{cult}_{img,x}$ is simultaneously induced to 
produce a higher quantity of culture-oriented labels but not a higher
specificity of culture-oriented labels.  Since effects on 
label composition were only appreciable for the first posterior image 
(see \textbf{Fig. 8}), we also
performed the specificity analysis restricting to labels from the first
posterior image (see \textbf{Fig. S1}).  But even when focusing on the first 
posterior image we see no significant increase in cultural specificity for 
$\textsc{cult}_{img,x}$.  Thus, we observe that priming effects on 
orientation of focus and label specificity do not necessarily correlate.

\paragraph{Similarity drives nuance.}
We observed already that treatments primed with the 
culture or ingredients image sets have less specificity compared to 
\textsc{ambg}.  We will now argue that the difference in specificity
can be explained by how ``similar'' a treatment's prior 
images are to the posterior images. The characterization of the content of an 
image is deeply complex, and has been approached from the perspectives of art\cite{panofsky1939studies,shatford1986analyzing},
psychology\cite{Tversky1977327}, and information retrieval\cite{Jaimes20002}.  Here, we will take the simplistic 
approach of defining the similarity between two sets of images as the number of
labels in common between both sets.

\dots Results to follow \dots

This matches the intuition that the ambiguous set is most similar to the 
posterior set in that they both feature prepared meals, while the ingredients
set is more similar to the posterior set than is the cultural set, since the
ingredients set at least features food prominently.  Although similarity 
between the cultural set and the posterior set could be argued on the basis
of cultural content, this content is less imageable, more abstract, and
and according to the theory of \cite{Jaimes20002} should require more 
knowledge to ascertain.

Under our measure of similarity, our results show that
more specific labels are provided to the posterior images when the prior and
posterior image sets are more similar.
An appealing explanation for 
this is the notion of \textit{negative priming}, wherein the participant
becomes desensitized to stimuli that do not seem salient [***].  Thus, after 
workers had been shown successive images of prepared meals in the ambiguous
prior set, they would be much less inclined to provide generic terms such as 
\texttt{food} or 
\texttt{meal} to images in the posterior set, since these terms would no 
longer seem salient in the 
context of their priming treatment.  

\paragraph{Priming for better HPU performance.}
Thus, if one seeks very nuanced sorting or labeling from human workers, 
we recommend assembling the HPUs in a hierarchical fashion, with the 
HPUs at the top of the hierarchy performing coarse sorting (or labeling), 
splitting objects into streams to be handled by HPUs performing more nuanced 
work.  To implement this for an
image-labeling task, one could first serve the images for coarse labeling,
then cluster the images based on the labels and a language model, and then 
re-serve the images, ensuring that each HPU operates on images from a single 
cluster.  The more ``homogeneous'' a cluster of images, the more one might gain
by serving the cluster for another round of labeling.

This exhibits an interesting difference between HPUs and CPUs.  In general,
whenever an aspect of a problem can be parallelized when employing CPUs, one
gains efficiency.  But in the case of hysteretic HPUs, we see an 
example where one might lose efficiency through parallelization.

\subsection*{Conclusion}
We demonstrated that in-task priming can have a strong effect.  
For an image labeling task, in-task priming was stronger in its effects on 
the orientation and specificity of labels than was the disclosure of funder 
names chosen to signal the requester's interest in specific content.
Employers on microtask platforms should 
beware that strong biases may be introduced by the very tasks they provide.
Further research into the nature of these effects is warranted.  At the 
same time, in-task priming might be exploited to improve task output, for 
example, to drive more (or less) nuanced responses.

We demonstrated that our algorithmic definition of priming, and its 
implementation as the measure $\theta_\text{NB}$ provide a sensitive,
general-purpose method for detecting priming.  Even when the expected effects 
on label orientation and specificity could not be detected, $\theta_\text{NB}$ 
was high.  $\theta_\text{NB}$ provides direct statistical evidence that 
populations have been affected by priming, even when the effects are not 
understood. Along with our proposed definition, we advocate a more general 
view of priming as a form of hysteresis.

\subsection*{Materials and Methods}

\textbf{AMT Tasks.} The task described in \textbf{Experimental Setup}
was served to 900 workers, who were uniformly randomly assigned to
treatments.  All treatments had at least 126 task completions.  To obtain 
uniformly sized treatments, we randomly selected 126 samples from each 
treatment. Workers were
only allowed to participate once and could not see the prior or posterior 
images, nor the funder names when previewing the task.

\textbf{Naive Bayes Classifier.}
The features provided to the Naive Bayes classifier consisted of the labels
that the worker provided along with the image and specific location in which 
the label was entered
(each image had five distinct label inputs).  Thus, \texttt{wine} occurring in 
position 1 of image 2 would be considered as a distinct feature from 
\texttt{wine} occurring in position 2 of image 2. The classifier 
had better performance when this distinction was made.

\textbf{Building the ontology of terms.}  The ontology was built manually
by the experimenters, starting from the corpus of labels. Labels occurring only 
once were discarded.  We created the root classes \texttt{activity}, 
\texttt{adjective}, \texttt{culture}, 
\texttt{food}, and \texttt{thing}. \texttt{culture} and \texttt{food} were 
given top-level status because they were prominently represented in the corpus.
Whenever one label was judged to be a more specific case of another, a
directed edge was drawn from the more general to the more specific. This
judgment call was made by the experimenters.  However, the labels from all 
treatments were aggregated before assembling the ontology,  to 
ensure that the ontology construction would not be affected by 
knowledge of the treatments from which labels originated. 
Labels which were judged to be roughly synonymous were treated as identical,
and labels judged to be misspellings were corrected. 

\textbf{Statistical significance for relative specificity.}
We used \textbf{Eq.~\ref{eq:specificity}} to compute the relative specificity
of two treatments.  To analyze relative specificity statistically,
we first develop \textbf{Eq.~\ref{eq:specificity}} from an intermediate 
definition.  Let the specificity of a worker $u$ relative to a 
\textit{basis treatment} $\mathcal{V}$, denoted $S_V(u)$, be the expected 
relative specificity between $u$ and a worker uniformly sampled from 
$\mathcal{V}$:

\begin{align}
	S_V(u) =  \text{E}_V\{s(u,V)\}, \quad V \in_R \mathcal{V},
		\label{eq:basis-specificity}
\end{align}
where $V$ is a random, uniformly sampled worker from $\mathcal{V}$.  We 
subscript the expectation operator with $V$ to indicate 
that the expectation is taken over the distribution of $V$.  Then, the 
relative specificity of two treatments, $S(\mathcal{U},{V})$, can be 
written as:
\begin{align}
	S(\mathcal{U},\mathcal{V}) = 
		\text{E}_U\{S_V(U)\}, \quad U \in_R \mathcal{U},
		\label{eq:treatment-specificity}
\end{align}
where $U$ is a random, uniformly sampled worker from $\mathcal{U}$.  This 
yields 
estimators for $S_V(u)$ and $S(\mathcal{U}, \mathcal{V})$, and their variances
as shown below (note that 
\textbf{Eq.~\ref{eq:treatment-specificity-estimator}} is
equivalent to \textbf{Eq.~\ref{eq:specificity}}):

\begin{align}
	\hat{S}_V(u) &= \frac{1}{|\mathcal{V}|} \sum_{v \in \mathcal{V}} s(u,v), 
		\label{eq:basis-specificity-estimator} \\
		\hat{S}(\mathcal{U},\mathcal{V}) 
		&= \frac{1}{|\mathcal{U}|} \sum_{u \in \mathcal{V}} \hat{S}_V(u),
		\label{eq:treatment-specificity-estimator} \\
		\text{Var}_V\{\hat{S}_V(u)\} 
			&= \frac{1}{|\mathcal{V}|}\text{Var}_V\{s(u,V)\},
		\label{eq:basis-specificity-variance} \\
		\text{Var}_{U,V}\{\hat{S}(\mathcal{U},\mathcal{V})\} 
			&= \frac{1}{|\mathcal{V}|\cdot|\mathcal{U}|^2} 
			\sum_{u\in U}\text{Var}_V\{{s}(u,V)\}.
		\label{eq:treatment-specificity-variance}
\end{align}
$\text{Var}_V\{s(u,V)\}$ can be directly calculated as a sample variance.
\textbf{Equations \ref{eq:basis-specificity-estimator}} through 
\textbf{\ref{eq:treatment-specificity-variance}} provide enough information to test the hypotheses:
\begin{align}
	\begin{matrix}
		\mathbf{H_0}: S(\mathcal{U}, \mathcal{V}) = 0 \\[0.5em]
		\mathbf{H_1}: S(\mathcal{U}, \mathcal{V}) \neq 0
	\end{matrix}
\end{align}

\section*{Acknowledgements}
\section*{References}
\begingroup
\renewcommand{\chapter}[2]{}
\bibliography{newbib.bib}
\endgroup
\bibliographystyle{plain} 

\section*{Figure Legends}

\paragraph{Table 1.}
	Experimental treatments.  Workers were uniformly randomly
	assigned to one of the treatments listed above.  The full funder 
	names used were ``The Global Foundation
	for Cultural Recognition'' and ``The National Foundation for 
	Nutritional Awareness''.  The ambiguous, cultural, and ingredients 
	prior image sets are shown in \textbf{Figs. 2}, \textbf{3}, and 
	\textbf{4}.

\paragraph{Figure 1.}
	Posterior image set. These images were presented to all workers after the 
	prior image set, in the order shown. 

\paragraph{Figure 2.}
	Ambiguous prior image set. These images were presented to workers from 
	certain treatments (see \textbf{Table 1}) in the order shown.  

\paragraph{Figure 3.}
	Cultural prior image set. These images were presented to workers from 
	certain treatments (see \textbf{Table 1}) in the order shown.  

\paragraph{Figure 4.}
	Ingredients prior image set. These images were presented to workers from 
	certain treatments (see \textbf{Table 1}) in the order shown.  

\paragraph{Figure 5.}
$F_1$ score and $\theta_\text{NB}$ for the various 
priming treatments, as measured using a Naive Bayes classifier. Dotted lines 
show the 95\% confidence interval
assuming the classifier performed no better than chance.  Each panel 
presents performance results for the binary classification between a basis 
treatment, indicated by the inset, and other treatments, indicated on the 
abscissa.

\paragraph{Figure 6.}
$F_1$ score and $\theta_\text{NB}$ for the classification of 
$\textsc{cult}_{img}$ and $\textsc{ingr}_{img}$ using a Naive Bayes 
classifier.  Dotted lines 
show the 95\% confidence interval
assuming the classifier performed no better than chance.  The classifier is provided only the labels attributed to
particular images as indicated on the abscissa.

\paragraph{Figure 7.}
Percent label composition (culture- vs food-oriented labels) for 
various images and treatments.  Panel A shows the 
composition of labels, aggregated over all treatments, as a function of 
posterior image.
Panel B shows the composition of labels, aggregated over all images, as
a function of treatment.  Panel C shows the composition of labels 
attributed to the first posterior image by various treatments.

\paragraph{Figure 8.}
Excess culture-oriented label composition in the labels provided
by $\textsc{cult}_{img}$ as compared to $\textsc{ambg}$, for various
posterior images (see \textbf{Eq. 1} for calculation).

\paragraph{Figure 9.}
Relative specificity of various treatments.
Each panel shows the comparison between a basis treatment (inset) and 
target treatments (abscissa).
Bar heights indicate relative specificity of the target 
treatment compared to the basis (a positive quantity means the target 
is more specific), and are normalized to units of standard 
deviations of the null comparison.  Dotted horizontal lines indicate
the 95\% confidence interval for the null comparison, which necessarily
has zero relative specificity (see \textbf{Methods} 
for an explanation of the null comparison).  Panels D, E, and F 
represent relative specificity calculated using only culture-oriented
labels, while panels G, H, and I using only food-oriented labels.
Labels that are both culture- and food-oriented are excluded from the
calculations for panels D through I.

\paragraph{Figure S1}
Relative specificity of various treatments, for labels attributed
to the first posterior image.
Each panel shows the comparison between a basis treatment (inset) and 
target treatments (abscissa).
Bar heights indicate relative specificity of the target 
treatment compared to the basis (a positive quantity means the target 
is more specific), and are normalized to units of standard 
deviations of the null comparison.  Dotted horizontal lines indicate
the 95\% confidence interval for the null comparison, which necessarily
has zero relative specificity (see \textbf{Methods} 
for an explanation of the null comparison).  Panels D, E, and F 
represent relative specificity calculated using only culture-oriented
labels, while panels G, H, and I using only food-oriented labels.
Labels that are both culture- and food-oriented are excluded from the
calculations for panels D through I.

\end{document}


